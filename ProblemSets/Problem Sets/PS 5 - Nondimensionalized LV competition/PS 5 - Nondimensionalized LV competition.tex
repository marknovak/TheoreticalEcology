\documentclass[10pt,A4]{article}
\usepackage[latin1]{inputenc}
\usepackage{amsmath}
\usepackage{amsfonts}
\usepackage{amssymb}
\usepackage{graphicx}
\author{Mark Novak}
\title{BI592 Theoretical Ecology \\ PS 5. Non-dimensional analysis of two-species competition}
\date{}
\author{}
\begin{document}
\maketitle

\section{Part A}
In class we studied the classic Lotka-Volttera competition model: 
\begin{equation}
\frac{dN_i}{dt}=r_i N_i\left(1-\frac{N_i - \alpha_{ij}N_j}{K_i}\right)
\end{equation}
for two species.  As we also showed in class, this model can be non-dimensionalized to:
\begin{eqnarray}
\frac{du_1}{dt}&=u_1(1-u_1-a_{12}u2)\\
\frac{du_2}{dt}&=\rho u_2(1-u_2-a_{21}u1),
\end{eqnarray}
where $u_i$ reflects the proportional abundance of species $i$, ($u_i =\tfrac{N_i}{Ki})$, $a_{ij}$ reflects the interspecific effect of species $j$ on species $i$ relative to $i$'s intraspecific effect on itself, $a_{ij} = \tfrac{\alpha_{ij}K_j}{K_i}$, and $\rho$ reflects the relative growth rate of the two species, $\rho=r_2/r_1$.

Using \emph{Mathematica}, determine the two zero-growth isoclines of this model.�

\section{Part B} 
Now switch to \emph{R} and plot these isoclines (using the \emph{curve} function) for the following four cases, choosing parameter values for�$a_{12}$ and $a_{21}$ accordingly:�
\begin{enumerate}
\item competitive dominance with species 1 being the superior competitor,
\item competitive dominance with species 2 being the superior competitor,
\item  species coexistence, and
\item  a priority effect, where coexistence or invasibility of species 2 depends on the abundance of the species 1 and vice versa.
\end{enumerate}

Note that after plotting the isocline of species 2 as a function of species 1, you will have to rearrange the equation for the isocline of species 1 in order to add it on the same plot.  Overlay the isocline plots on vector-fields by first using the plotVectorField function of the 'VectorField.R' script (uploaded to the website).

Confirm the results of your four phase diagrams by simulating the dynamics of the four cases to produce five additional plots of population size versus time (1 plot for cases 1-3, and two plots for case 4 with different starting population sizes but the same parameter values).� Assume $\rho=1$ for all simulations.

What do your graphs indicate about how the relative strengths of inter- and intra-specific competition determine the final species abundances?� That is, what must the relative strengths of inter- and intra-specific competition be to obtain a given type of equilibrium?


\end{document}