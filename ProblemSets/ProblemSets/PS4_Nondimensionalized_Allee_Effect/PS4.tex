\documentclass[11pt,letterpaper]{article}

\usepackage[margin=1in]{geometry}
\usepackage[latin1]{inputenc}
\usepackage{amsmath, amsfonts, amssymb}
\usepackage{graphicx}
\usepackage{hyperref} % enable links within pdf
\hypersetup{colorlinks = true, linkcolor = black, urlcolor = blue}

\setcounter{secnumdepth}{0}  % don't number sections (stars not needed)

\title{Problem Set 4\\Logistic with Allee effect\vspace{-3em}}
\date{}
\author{}

\begin{document}
\maketitle

In the logistic model,
\begin{equation}
\frac{dN}{dt}=rN\left(1-\frac{N}{K}\right),
\end{equation}
a population always increases when rare.� However, sometimes a population will decline when rare, a phenomenon called an Allee effect.� This occurs when its per capita growth rate is positively correlated with its population size below some critical threshold.� In abalone, for example, the probability of finding a mate declines below a critical density of individuals.  An Allee effect can be incorporated into the logistic model as follows: 
\begin{equation}
\label{Allee}
\frac{dN}{dt}=rN\left(1-\frac{N}{K}\right)\left(\frac{N}{K}-\frac{A}{K}\right),
\end{equation}
This Allee model has 1 state variable $N(t)$, and 3 parameters ($r$, $K$, and $A$). Time ($t$) is in some sense implicitly included since we are modeling population dynamics as a differential (instantaneous) process.� The units of these variables and parameters are:
\begin{itemize}
\item	$N(t)$ - Number of individuals at time $t$
\item	$K$ - Number of individuals
\item	$A$ - Number of individuals	
\item	$r$ - $1/t$		
\item	$t$ - time
\end{itemize}


We can non-dimensionalize this Allee model by scaling the state variable and grouping the parameters as 
\begin{equation}
x:=\frac{N}{K},\quad \tau:=rt, \quad \text{ and } \quad \beta:=\frac{A}{K}.
\end{equation}
$x$ is now a dimensionless variable (i.e., has no units) and represents the proportional population size (proportional to the population's carrying capacity). $\beta$ is a dimensionless parameter and represents the minimum proportion that is needed for positive growth. $\tau$ is a dimensionless reflection of time (``generation time'').

By substituting these scaled variables and parameters the model becomes
\begin{equation}
\frac{d\left(\tfrac{N}{K}\right)}{r \cdot dt}=\frac{r}{r}\frac{N}{K}\left(1-\frac{N}{K}\right)\left(\frac{N}{K}-\frac{A}{K}\right),
\end{equation}
which reduces to
\begin{equation}
\label{NonDimAllee}
\frac{dx}{d\tau}=x(1-x)(x-\beta).
\end{equation}


\section{Part A}
Using \emph{Mathematica}:
\begin{enumerate}
\item Determine the equilibria of both Allee effecs models:  Model 1 (eqn. \ref{Allee}) and Model 2 (eqn. \ref{NonDimAllee}).

\item Assess the local stability properties of the equilibria for the two models.

\item Explain why the population declines to extinction once the population size falls below the critical threshold of $A$ in Model 1 (eqn. \ref{Allee}).
\end{enumerate}

\section{Part B}
Now switch to R.  Using the \emph{deSolve} package, simulate and plot the dynamics of both models using equivalent parameter sets at two starting population sizes:� one above the threshold and one below the threshold.� Use any reasonable parameter values you'd like.� Produce a figure with all four plots arranged on a grid (i.e., 2 columns, 2 rows).�

You may want to refer to the \emph{`StabilityLogisticGrowth.nb'} and \emph{`LogisticGrowthExercises.R'} codes from our previous in-class exercises. (Try using \emph{Mathematica} to perform Part B too, if you'd like!)

\end{document}