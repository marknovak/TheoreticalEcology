\documentclass{article}
\usepackage{setspace}
\usepackage[text={6.5in,8.5in},centering]{geometry}
\geometry{verbose,a4paper,tmargin=2.4cm,bmargin=2.4cm,lmargin=2.4cm,rmargin=2.4cm}
\usepackage{graphicx,amsmath,cases,multirow,appendix,graphicx,xcolor}

\setlength\parindent{0pt}

\newcommand{\note}[1]{\colorbox{gray!30}{#1}}
\newcommand{\ind}{\-\hspace{1cm}}

\begin{document}

\noindent\makebox[\textwidth][c]{\Large\bfseries Quiz 1 - Monod's Nightmare}

\rule[0.5ex]{\linewidth}{1pt}

Under ideal growing conditions in the laboratory, the population size of the common bacterium \emph{Escherichia coli} can double in around 20 minutes.  \emph{E. coli}'s cells are rod-shaped and are roughly approximated by a rectangular box that is 2 $\mu m$ long, 1 $\mu m$ wide, and 1 $\mu m$ high.  (1 $\mu m$ = 1 micro-meter = 1 x 10-6 meters = 0.000001 m).  For comparison, a strand of human hair is roughly 100 $\mu m$ wide.

\vspace{1cm}

(a) Would a continuous-time model (the solution for which is $N_t=N_0 e^{rt}$) or a discrete-time model (the solution for which is $N_t = \lambda^t N_0$) be more appropriate for describing this population?  Why?

\vspace{1cm}

\note{Continuous.  Reproduction occurs continuously, not in discrete synchronous bouts.}

\vspace{1cm}

(b)  What is \emph{E. coli}'s growth rate $r$ (in minutes) under these ideal growing conditions.  Express your answer by solving for the equation you would use to calculate $r$ to the simplest solution possible.

\vspace{1cm}
\begin{align*}
	t=20 \; \text{minutes;} &\;\;\; N_t= 2 N_0\\
	N_t&=N_0 e^{rt}\\
	ln\left(\frac{N_t}{N_0}\right)&=rt\\
	r&=\frac{ln\left(\frac{N_t}{N_0}\right)}{t}=\frac{ln(2)}{20} \approx \frac{0.693}{20} \approx 0.035 \% \text{per minute}
\end{align*}

\vspace{1cm}
(c) Let's assume that our classroom is roughly 10 $m$ long, 10 $m$ wide, and 8 $m$ high (it’s not, I just made up these numbers).  How long it would take for an exponentially growing population of \emph{E. coli} under ideal conditions to fill our empty classroom when starting from a single individual bacterium?

\vspace{1cm}
\note{Volume of each \emph{E.coli}:}\\
\ind $1\mu m \cdot 1\mu m \cdot 2\mu m = 2\mu m^3$ \\
\ind $ 1\cdot 10^{-6}m \cdot 1\cdot 10^{-6}m \cdot 2\cdot 10^{-6}m = 2 \cdot 10^{-18}m^3$\\
\note{Volume of room:}\\
\ind $10 m \cdot 10 m \cdot 8 m = 800m^3$\\
\note{Therefore, number of \emph{E. coli} needed to fill room:}\\
\begin{equation*}
	\frac{800 m^3}{2 \cdot 10^{-18} m^3}\approx 4 \cdot 10^{20}
\end{equation*}
\note{Thus:}\\
\begin{align*}
	N_0&=1\\
	N_t=N_0 e^{rt}\\
	ln\left(\frac{N_t}{N_0}\right)&=rt\\
	t&=\frac{ln \left(\frac{4\cdot 10^{20}}{1}\right)}{t}=\frac{ln \left(\frac{4\cdot 10^{20}}{1}\right)}{\frac{ln(2)}{20}} \approx \frac{47.438}{0.035} \approx 1369 \; \text{minutes} \approx 22.8 \; \text{hours}
\end{align*}



\end{document}