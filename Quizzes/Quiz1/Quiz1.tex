\documentclass{article}
\usepackage{setspace}
\usepackage[text={6.5in,8.5in},centering]{geometry}
\geometry{verbose,a4paper,tmargin=2.4cm,bmargin=2.4cm,lmargin=2.4cm,rmargin=2.4cm}
\usepackage{graphicx,amsmath,cases,multirow,appendix,graphicx,xcolor}

\setlength\parindent{0pt}

\newcommand{\note}[1]{\colorbox{gray!30}{#1}}
\newcommand{\ind}{\-\hspace{1cm}}
\newcommand*{\blanks}[1][4em]{\rule{#1}{.4pt}}

\begin{document}

\noindent\makebox[\textwidth][c]{\Large\bfseries Quiz 1 - Monod's Nightmare}

\rule[0.5ex]{\linewidth}{1pt}
\begin{center}
	\textbf{My name is:} \blanks[150pt]
\end{center}
\rule[0.5ex]{\linewidth}{1pt}

Under ideal growing conditions in the laboratory, the population size of the common bacterium \emph{Escherichia coli} can double in around 20 minutes.  \emph{E. coli}'s cells are rod-shaped and are roughly approximated by a rectangular box that is 2 $\mu m$ long, 1 $\mu m$ wide, and 1 $\mu m$ high.  (1 $\mu m$ = 1 micro-meter = 1 x 10-6 meters = 0.000001 m).  For comparison, a strand of human hair is roughly 100 $\mu m$ wide.

\vspace{1cm}

(a) Would a continuous-time model (the solution for which is $N_t=N_0 e^{rt}$) or a discrete-time model (the solution for which is $N_t = \lambda^t N_0$) be more appropriate for describing this population?  Why?

\vspace{3cm}


\vspace{1cm}

(b)  What is \emph{E. coli}'s growth rate $r$ (in minutes) under these ideal growing conditions.  Express your answer by solving for the equation you would use to calculate $r$ to the simplest solution possible.

\vspace{4cm}

\vspace{1cm}

(c) Let's assume that our classroom is roughly 10 $m$ long, 10 $m$ wide, and 8 $m$ high (it’s not, I just made up these numbers).  How long it would take for an exponentially growing population of \emph{E. coli} under ideal conditions to fill our empty classroom when starting from a single individual bacterium?

\vspace{1cm}



\end{document}