\documentclass{article}
\usepackage{setspace}
\usepackage[text={6.5in,8.5in},centering]{geometry}
\geometry{verbose,a4paper,tmargin=2.4cm,bmargin=2.4cm,lmargin=2.4cm,rmargin=2.4cm}
\usepackage{graphicx,amsmath,cases,multirow,appendix,graphicx,xcolor}

\setlength\parindent{0pt}

\newcommand{\note}[1]{\colorbox{gray!30}{#1}}
\newcommand{\ind}{\-\hspace{1cm}}
\newcommand*{\blanks}[1][4em]{\rule{#1}{.4pt}}


\begin{document}

\noindent\makebox[\textwidth][c]{\Large\bfseries Quiz 2 - MSY Harvest Scenarios}

\rule[0.5ex]{\linewidth}{1pt}
\begin{center}
	\textbf{My name is:} \blanks[150pt]
\end{center}
\rule[0.5ex]{\linewidth}{1pt}

Assume that the curve (solid line) of the following figure reflects the underlying mean empirical relationship between the population growth rate ($dN/dt$) of a commercially-harvested fish species and its abundance ($N$).  Such a density-dependent relationship is often called the stock-recruitment function in fisheries management.  The labeled lines (\textbf{A} and \textbf{B}) illustrate two contrasting harvesting strategies that could be applied to this fishery.
\begin{center}
\includegraphics[width=6cm]{figs/image}\\
\end{center}
(a) Based on the slopes of the two lines, match these strategies to the following models describing the relationship between the population's recruitment rate ($dN/dt$) and its stock size ($N$), intrinsic growth rate ($r$), carrying capacity ($K$), and the rates of harvest ($h$ or $H$).
\begin{align*}
	\frac{dN}{dt}=rN \left(1-\frac{N}{K}\right)-hN &\quad \quad \quad &   \frac{dN}{dt}=rN \left(1-\frac{N}{K}\right)-H \\
	\text{Scenario: }\blanks & \quad \quad \quad & \text{Scenario: }\blanks
\end{align*}

\vspace{1cm}

(b) Describe these strategies \emph{in less than} 15 words each.\\
\ind Scenario A:\\
\vspace{1cm}

\ind Scenario B:\\
\vspace{1cm}

(c) Illustrate on the figure how the line for scenario \textbf{A} would change if harvesting pressure were to increase under this scenario.  Illustrate an increase for scenario \textbf{B}. \\

(d) Assuming that the stock-recruitment relationship depicted in the figure is indeed an adequate description of the empirical situation for this fishery, use your answer to the previous question to justify why one of the two scenarios (\textbf{A} or \textbf{B}) might be considered a more sustainable harvesting scenario than the other. 

\end{document}