\documentclass{article}
\usepackage{setspace}
\usepackage[text={6.5in,8.5in},centering]{geometry}
\geometry{verbose,a4paper,tmargin=2.4cm,bmargin=2.4cm,lmargin=2.4cm,rmargin=2.4cm}
\usepackage{graphicx,amsmath,cases,multirow,appendix,graphicx,xcolor}

\setlength\parindent{0pt}

\newcommand{\note}[1]{\colorbox{gray!30}{#1}}
\newcommand{\ind}{\-\hspace{1cm}}
\newcommand*{\blanks}[1][4em]{\rule{#1}{.4pt}}


\begin{document}

\noindent\makebox[\textwidth][c]{\Large\bfseries Quiz 6 - Interaction Strengths}

\rule[0.5ex]{\linewidth}{1pt}
\begin{center}
	\textbf{My name is:} \blanks[150pt]
\end{center}
\rule[0.5ex]{\linewidth}{1pt}

One of the simplest non-pathological models of a predator-prey interactions adds only logistic prey growth to the classic (pathological) Lotka-Volterra model:
\begin{equation*}
	\frac{dR}{dt} = b R \left ( 1 - \frac{R}{K} \right ) - a R C 
	\quad \quad \quad
	\frac{dC}{dt} = e a R C - d C \, .	
\end{equation*}
An empirical ecologist measuring the effect of a predator on a prey species might measure values reflecting any of the following terms: $a$, $a R$, $a C$, and $aRC$.  Indeed, all four terms could be (and have been) called measures of the ``interaction strength'' between the two species (i.e. the top-down effect of the predator on the prey).  But clearly tthey are not equivalent.


1) \textit{Why} aren't these measures equivalent?  Answer by interpreting each  for an empiricist and by defining it in terms of its units.

\vspace{0.1 cm}

$a$ -- 
%\vspace{1.5 cm}
	\textcolor{red}{
		per capita attack rate; encounter rate; effect of one predator individual on one prey individual.\\
		Units: number of prey (eaten) per predator per prey (available) per time.
	}
	\vspace{0.1 cm}

$aR$ --
%\vspace{1.5 cm}
	\textcolor{red}{
		rate at which individual predators feed on prey population; functional response; effect of prey population on one predator individual.\\
		Units: number of prey (eaten) per predator per time.
}
\vspace{0.1 cm}

$aC$ --
%\vspace{1.5 cm}
	\textcolor{red}{
		rate at which individual prey are consumed by predator population; individual prey vulnerability.\\
		Units: number of prey (eaten) per prey (available) per time.
}
\vspace{0.1 cm}

$aRC$ --
%\vspace{1.5 cm}
	\textcolor{red}{
		rate at which prey are consumed by predator population.\\
		Units: total number of prey (eaten) per time.
}
\vspace{1 cm}


%\vspace{1 cm}
2) Which measure(s) would an empirical ecologist prefer to measure if they want to compare the ``interaction strength'' that different predator species have on one common prey species?  Why?

%\vspace{3 cm}
\vspace{0.1 cm}
\textcolor{red}{
	Could use either $a$ or $aR$, since $R$ would be constant across all comparisons. Either measure standardizes by predator population size, allowing the effects of different predator species with different population sizes be contrasted on a common basis. 
}
\vspace{1 cm}


3) Which measure(s) would they best measure if they want to compare different pairs of predator-prey species across an entire food web?  Why?

%\vspace{3 cm}
\vspace{0.1 cm}
\textcolor{red}{
	Use $a$, since it standardizes the effect-measure by both prey and predator population size, allowing the effects between different predator-prey pairs with different populations sizes to be contrasted on a common basis.
}
\vspace{1 cm}


3) Which measures(s) would a theoretical ecologist hoping to parameterize their mathematical models with empirical data prefer to have the empiricist estimate?  Why?

%\vspace{3 cm}
\vspace{0.1 cm}
\textcolor{red}{
	Would prefer $a$.  Both $R$ and $C$ are dynamic state variables, whereas $a$ is a fixed-valued parameter.  Measures of $a R$, $a C$, or $a C R$ would not allow them to parameterize their simulations (as easily). 
}


\end{document}